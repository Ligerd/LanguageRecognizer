\documentclass[8pt]{article}
\usepackage[utf8]{inputenc}
\usepackage{polski}
\usepackage{geometry}
\usepackage[argument]{graphicx}
\usepackage{graphicx}
\graphicspath{{images/}{images2/}}


\usepackage{fancyhdr}
\usepackage{lastpage}

\pagestyle{fancy}
\fancyhf{}
\rfoot{Strona \thepage \hspace{1pt} z \pageref{LastPage}}
\lhead{Specyfikacja funkcjonalna systemu z siecią neuronową do rozpoznawania języka dokumentu}
\begin{document}
	\begin{titlepage}
		\begin{center}
			\Large
			Politechnika Warszawska 
			
			Wydział Elektryczny 
			
			Kierunek Informatyka 
			\vfill
			\Huge \textsc{Specyfikacja funkcjonalna systemu z siecią neuronową do rozpoznawania języka dokumentu}
		\end{center}
		\vfill
		
		
		
		\begin{center}
			\Large Wykonał: Aliaksandr Karolik 
		\end{center}
		\begin{center}
			\Large	Warszawa, 23.03.2019
		\end{center}
	
		
		
	\end{titlepage}
	\newpage
	\Large\tableofcontents
	
	
	\newpage

\section{Teoretyczny wstęp}
\hspace*{1 cm} Głównym celem programu jest wczytanie z pliku wejściowego analizowanego tekstu oraz za pomocą sztucznej sieci neuronowej rozpoznanie języka wczytanego tekstu.\newline
\hspace*{1 cm} Sieć neuronowa (sztuczna sieć neuronowa) – ogólna
nazwa struktur matematycznych i ich programowych
lub sprzętowych modeli, realizujących obliczenia lub
przetwarzanie sygnałów poprzez rzędy elementów,
zwanych sztucznymi neuronami, wykonujących
pewną podstawową operację na swoim wejściu.
Oryginalną inspiracją takiej struktury była budowa
naturalnych neuronów oraz układów nerwowych, w
szczególności mózgu. \newline
\hspace*{1 cm} Cechą wspólną wszystkich sieci neuronowych
jest to, że na ich strukturę składają się neurony
połączone ze sobą synapsami. Z synapsami
związane są wagi, czyli wartości liczbowe,
których interpretacja zależy od modelu.
\subsection{Schemat sztucznego neuronu}
\hspace*{1 cm}Do wejść neuronów doprowadzane są sygnały dochodzące z wejść sieci lub neuronów warstwy
poprzedniej. Każdy sygnał mnożony jest przez odpowiadającą mu wartość liczbową
zwaną wagą. Wpływa ona na percepcję danego sygnału wejściowego i jego udział w
tworzeniu sygnału wyjściowego przez neuron. \newline
\hspace*{1 cm}Waga może być pobudzająca -- dodatnia lub opóźniająca -- ujemna.
Jeżeli nie ma połączenia między neuronami, to waga jest równa zero. Zsumowane iloczyny sygnałów i wag stanowią argument funkcji aktywacji neuronu.\newpage
\subsection{Formuła opisująca działanie neuronu}
\begin{equation}
y = f(s)  
\end{equation}
\hspace*{1 cm} Gdzie:
\begin{equation}
s=\sum \limits_{i=0}^{n} x_i  w_i 
\end{equation}
\hspace*{1 cm} Funkcja aktywacji może przyjmować różną postać w zależności od konkretnego modelu neuronu.
\newline \hspace*{1 cm} Wymagane cechy funkcji aktywacji  to:
\begin{itemize}
\item Ciałe przejście pomiędzy swoją wartością maksymalną a minimalną;
\item Łatwa do obliczenia i ciągła pochodna.
\end{itemize}
\section{Wymagania funkcjonalne}
\hspace*{1 cm} Wymaganiami funkcjonalnymi programu są: 
\begin{itemize}
\item wczytanie analizowanego tekstu;
\item rozpoznawanie języka analizowanego tekstu;
\item możliwość uczenia sztucznej sieci neuronowej;
\item zapisanie wag połączeń pomiędzy wyjściem neuronu o numerze \texttt{m} a wejściem neuronu o numerze \texttt{n} z kolejnej warstwy;
\item wczytanie z pliku wag  połączeń pomiędzy wyjściem neuronu o numerze \texttt{m} a wejściem neuronu o numerze \texttt{n}  z kolejnej warstwy.
\end{itemize}
\section{Przykładowy komunikat o błędach}
\hspace*{1 cm} Błędy występujące w programie, przykładowo plik zawiera niepoprawne dane, będą wyświetlane w postaci okienek pop-up wraz odpowiednią 
ą opisującą zaistniały błąd.\newline

\begin{figure}[h]
\centering{\includegraphics[scale=1.2]{error1.png}}
\caption{Przykładowy komunikat o błędzie}
\label{fig:image}
\end{figure}
\section{Jak korzystać z programu}
\hspace*{1 cm} Program nie będzie interaktywny. Wszystkie ustawienia programu podawane są jako argumenty. Program ma możliwość działania ~~~~~ w dwóch trybach: 
\begin{enumerate}
\item Tryb uczenia sieci neuronowej. W wybranym trybie program będzie analizować  zestaw danych podanych w pliku wejściowym przez użytkownika. Na podstawie danych zawartych w pliku wejściowym program będzie dobierać i zmieniać wagi połączeń pomiędzy wyjściem neuronu o numerze m a wejściem neuronu o numerze n z kolejnej warstwy. Prykładowe wywołanie programu: \newline 
\begin{verbatim}
python main.py -t -f plik_wejściowy.txt -i liczba_Iteracji
\end{verbatim}
Opis podanych argumentów:
\begin{itemize}
\item Argument \texttt{-t} decyduje o tym, że program musi działać w trybie ucznenia sieci;
\item Argument \texttt{-f} mówi o tym że następnym argumentem będzie  plik wejściowy;
\item Argument \texttt{-i} mówi o tym że następnym argumentem będzie liczba iteracji dla uczenia sieci neuronowej.
\end{itemize}
\item Tryb rozpoznawania języka tekstu. W wybranym trybie program będzie próbować rozpoznać, w którym języku jest napisany tekst w pliku wejściowym. Przykładowe wywołanie programu:
\begin{verbatim}
python main.py -p -f plik_wejściowy.txt 
\end{verbatim}
Opis podanych argumentów:
\begin{itemize}
\item Argument \texttt{-p} decyduje o tym że program muszi działać w trybie rozpoznawania języka;
\item Argument \texttt{-f} mówi o tym że następnym argumentem będzie  plik wejściowy.
\end{itemize}
\end{enumerate}
\section{Testy akceptacyjne}
\hspace*{1 cm}Testy akceptacyjne będą sprawdzać całą wymaganą funkcjonalność programu. Testy będą reprezentowały się następująco: 
\begin{itemize}
	\item Plik zawiera niepoprawne dane (np. same liczby).\newline
	Program powinien wyświetlić okienko z komunikatem o następującej treści:
	\begin{verbatim}
	Wystąpił błąd! Plik nazwa_pliku.txt zawiera niepoprawne dane!
	\end{verbatim}
	\item Program nie może otworzyć pliku podanego na wejście. \newline 
	Program powinien wyświetlić okienko z komunkatem o następującej treści:
	\begin{verbatim}
	Wystąpił błąd! Pliku o nazwa_pliku.txt nie udało się otwrozyć!
	\end{verbatim}
	\item Podczas próby zapisania wag połączeń pomiędzy neuronami wystąpił błąd. \newline
	Program powinien wyświetlić okienko z komunkatem o następującej treści:
	\begin{verbatim}
	Wystąpił błąd! Podczas próby zapisania wag neuronów.
	 Proszę o sprawdzenie uprawień.
	\end{verbatim}
	\item Nie udało się wczytać wagi połączeń między neuronami.\newline
	Program powinien wyświetlić okienko z komunkatem o następującej treści:
	\begin{verbatim}
	Wystąpił błąd! Podczas wczytywania wag połączeń między neuronami.
	\end{verbatim}
\end{itemize}
\end{document}
